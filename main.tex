% a mashup of hipstercv, friggeri and twenty cv
% https://www.latextemplates.com/template/twenty-seconds-resumecv
% https://www.latextemplates.com/template/friggeri-resume-cv

\documentclass[pastel]{simplehipstercv}
% available options are: darkhipster, lighthipster, pastel, allblack, grey, verylight, withoutsidebar
% withoutsidebar
\usepackage[utf8]{inputenc}
\usepackage[default]{raleway}
\usepackage[margin=1cm, a4paper]{geometry}
%\usepackage[hidelinks]{hyperref}
\hypersetup{hidelinks}

%------------------------------------------------------------------ Variablen

\newlength{\rightcolwidth}
\newlength{\leftcolwidth}
\setlength{\leftcolwidth}{0.23\textwidth}
\setlength{\rightcolwidth}{0.75\textwidth}

%------------------------------------------------------------------
\title{New Simple CV}
\author{\LaTeX{} Ninja}
\date{June 2019}

\pagestyle{empty}
\begin{document}


\thispagestyle{empty}
%-------------------------------------------------------------

\section*{Start}

\simpleheader{headercolour}{Virgile}{Daugé}{PhD Student}{white}



%------------------------------------------------

% this has to be here so the paracols starts..
\subsection*{}
\vspace{4em}

\setlength{\columnsep}{1.5cm}
\columnratio{0.23}[0.75]
\begin{paracol}{2}
\hbadness5000
%\backgroundcolor{c[1]}[rgb]{1,1,0.8} % cream yellow for column-1 %\backgroundcolor{g}[rgb]{0.8,1,1} % \backgroundcolor{l}[rgb]{0,0,0.7} % dark blue for left margin

\paracolbackgroundoptions

% 0.9,0.9,0.9 -- 0.8,0.8,0.8


\footnotesize
{\setasidefontcolour
\flushright
\begin{center}
    \roundpic{virgile}
\end{center}

\bg{cvgreen}{white}{About me}\\[0.5em]

{\footnotesize
Jeune chercheur en robotique, je suis curieux et j'aime concevoir, construire. La réalisation de solutions complètes à des problèmatiques concrètes me permet de vivre pleinement ma recherche. J'affectione particulièrement l'enseignement, car il prépare l'avenir.}
\bigskip

\bigskip

\bg{cvgreen}{white}{Domaines de prédilection} \\[0.5em]

Robotique ~•~ Enseignement ~•~ IA

\bigskip



\bigskip


\bg{cvgreen}{white}{Intérêts}\\[0.5em]

\texttt{Jeux de plateau} ~/~ \texttt{ vidéos}

\texttt{Théatre} ~/~ \texttt{zythologie}

\texttt{Linux} ~/~ \texttt{Python} ~/~ \texttt{Org-mode}

\vspace{4em}

\infobubble{\faLinkedin}{cvgreen}{white}{virgiledauge}
\infobubble{\faGithub}{cvgreen}{white}{virgiletn}

\phantom{turn the page}

\phantom{turn the page}
}
%-----------------------------------------------------------
\switchcolumn

\small
\section*{Curriculum}

\begin{tabular}{r| p{0.5\textwidth} c}
    \cvevent{2017--2021}{PhD}{LORIA/INRIA}{Nancy \color{cvred}}{Création et implémentation d'algorithmes novateurs pour l'exploration autonome d'environnements inconnus. Expérimentations pour validation des approches choisies dans un contexte réaliste.}{} \\
    \cvevent{2013--2016}{Etudiant ingénieur}{Télécom}{Nancy \color{cvred}}{Spécialisation dans le logiciel embarqué, stages au Canada chez \textit{Parallel Geometry} ou j'ai travaillé sur la norme AMF pour l'impression 3D. Ainsi qu'à INRIA où j'ai réalisé un controlleur réactif pour hexapod.}{} \\
    \cvevent{2011-2013}{DUT GEII}{IUT montet}{Nancy \color{cvred}}{Premiers pas dans la robotique, à travers des projets comme la réalisation d'un robot pour la coupe de france de robotique.}{} \\
\end{tabular}
\vspace{3em}

%% \begin{minipage}[t]{0.35\textwidth}
%% \section*{Degrees}
%% \begin{tabular}{r p{0.6\textwidth} c}
%%     \cvdegree{1710}{Captain}{Certified}{Tortuga Uni \color{headerblue}}{}{disney.png} \\
%%     \cvdegree{1715}{Bucaneering}{M.A.}{London \color{headerblue}}{}{medal.jpeg} \\
%%     \cvdegree{1720}{Bucaneering}{B.A.}{London \color{headerblue}}{}{medal.jpeg}
%% \end{tabular}
%% \end{minipage}


\begin{minipage}[t]{0.3\textwidth}
\section*{Publications}
\begin{tabular}{>{\footnotesize\bfseries}r >{\footnotesize}p{0.7\textwidth}}
    2019 & \emph{NAPS: a Nomadic and Accurate Positioning System}, Aerial Swarms | IROS 2019. \\
\end{tabular}
\bigskip

\section*{Talks}
\begin{tabular}{>{\footnotesize\bfseries}r >{\footnotesize}p{0.6\textwidth}}
  2019 & ``NAPS paper presentation'', \emph{IROS} in Macao.
\end{tabular}
\begin{tabular}{>{\footnotesize\bfseries}r >{\footnotesize}p{0.6\textwidth}}
  2018 & ``Robots en détresse ! '', \emph{Pint of science} in Nancy.
\end{tabular}
\end{minipage}
\hfill
\begin{minipage}[t]{0.3\textwidth}
\section*{Programming}
\begin{tabular}{r @{\hspace{0.5em}}l}
     \bg{skilllabelcolour}{iconcolour}{Org-mode} &  \barrule{0.3}{0.5em}{cvpurple}\\
     \bg{skilllabelcolour}{iconcolour}{Python} & \barrule{0.55}{0.5em}{cvgreen} \\
     \bg{skilllabelcolour}{iconcolour}{ROS/ROS2} & \barrule{0.5}{0.5em}{cvpurple} \\
     \bg{skilllabelcolour}{iconcolour}{Robotics} & \barrule{0.4}{0.5em}{cvpurple} \\
     \bg{skilllabelcolour}{iconcolour}{Linux} & \barrule{0.5}{0.5em}{cvpurple} \\
     \bg{skilllabelcolour}{iconcolour}{c/c++} &  \barrule{0.2}{0.5em}{cvpurple}\\
\end{tabular}
\end{minipage}

\section*{Enseignements}
\begin{tabular}{r| p{0.5\textwidth} c}
    \cvevent{64H}{Algorithmique et projets}{L2-L3}{DI FST Nancy \color{cvred}}{Principalement des cours d’algorithmique mais également du
  suivi et évaluation de projets étudiants, avec comme langage support le C. Réalisation
  d'un outil d'aide à la correction pour le TPs Notés de C. Ces
  différents enseignements comprenaient des TD, TP, TP Notés,
  surveillances et corrections d'examens.}{} \\
    \cvevent{96H}{Principes fondamentaux des systèmes informatiques}{L3}{Télécom Nancy \color{cvred}}{Accompagnement des étudiants dans la
  compréhension des Principes fondamentaux des systèmes informatiques
  en partant de l'algèbre de Boole pour arriver à de la programmation
  en assembleur (RISK simplifié). Comprenant des TD, TP, TP Notés,
  préparation d'examen et corrections de copies.}{} \\
    \cvevent{}{Encadrements projets et stages}{M1-M2}{LORIA \color{cvred}}{Initiation à la recherche et stages ingénieurs, proposition des sujets, suivi, évaluation.}{} \\
\end{tabular}
\vspace{2.5em}

%% \begin{minipage}[t]{0.3\textwidth}
%% \section*{Certificates \& Grants}
%% \begin{tabular}{>{\footnotesize\bfseries}r >{\footnotesize}p{0.55\textwidth}}
%%     1708 & Captain's Certificates \\
%%     1710 & Travel grant \\
%%     1715--1716 & Grant from the Pirate's Company
%% \end{tabular}
%% \bigskip

%% \section*{Languages}
%% \begin{tabular}{l | l}
%% \textbf{French} & {\phantom{x}\footnotesize mother tongue} \\
%% \textbf{English} & \pictofraction{\faCircle}{cvgreen}{4}{black!30}{1}{\tiny} \\
%% \textbf{Spanish}  & \pictofraction{\faCircle}{cvgreen}{1}{black!30}{3}{\tiny} \\
%% \textbf{Italian}  & \pictofraction{\faCircle}{cvgreen}{3}{black!30}{1}{\tiny}
%% \end{tabular}
%% \bigskip

%% \end{minipage}\hfill
%% \begin{minipage}[t]{0.3\textwidth}
%% \section*{Publications}
%% \begin{tabular}{>{\footnotesize\bfseries}r >{\footnotesize}p{0.7\textwidth}}
%%     1729 & \emph{How I almost got killed by Lady Swan}, Tortuga Printing Press. \\
%%     1720 & ``Privateering for Beginners'', in: \emph{The Pragmatic Pirate} (1/1720).
%% \end{tabular}
%% \bigskip

%% \section*{Talks}
%% \begin{tabular}{>{\footnotesize\bfseries}r >{\footnotesize}p{0.6\textwidth}}
%%   Nov. 1726 & ``Robots en détresse ! '', at: \emph{Pint of science} in Nancy.
%% \end{tabular}
%% \end{minipage}






\vfill{} % Whitespace before final footer

%----------------------------------------------------------------------------------------
%	FINAL FOOTER
%----------------------------------------------------------------------------------------
\setlength{\parindent}{0pt}
\begin{minipage}[t]{\rightcolwidth}
\begin{center}\fontfamily{\sfdefault}\selectfont \color{black!70}
{\small Virgile Daugé \icon{\faEnvelopeO}{cvgreen}{} 27 Rue Christian Pfister \icon{\faMapMarker}{cvgreen}{} Nancy \icon{\faPhone}{cvgreen}{} +33 6 99 09 02 29 \newline\icon{\faAt}{cvgreen}{} \href{mailto:virgile.dauge@loria.fr}{virgile.dauge@loria.fr}
}
\end{center}
\end{minipage}

\end{paracol}

\end{document}
